\section{Reálná čísla. Věta o supremu.}
\subsection{Číselné množiny}

\( \mathbb{N} = \left\{1, 2, 3, 4, 5, \dots \right\} \) - množina všech přirozených čísel \\
\( \mathbb{Z} = \left\{\dots, -3, -2, -1, 0, 1, 2, 3, \dots \right\} \) - množina všech celých čísel \\
\( \mathbb{Q} = \left\{\frac{p}{q} : p, q \in \mathbb{Z} \land q \neq 0 \right\} \) - množina všech racionálních čísel \\
\( \mathbb{R} = \left\{\dots \right\} \) - množina všech reálných čísel \\
\( \mathbb{R}^+ = \left\{x \in \mathbb{R} : x > 0 \right\} \) - množina všech kladných reálných čísel \\
\( \mathbb{R}^- = \left\{x \in \mathbb{R} : x < 0 \right\} \) - množina všech záporných reálných čísel \\
\( \mathbb{R} \setminus \mathbb{Q} \) - množina všech iracionálních čísel \\
\( \mathbb{R}^* = \mathbb{R} \cup \left\{+\infty, -\infty \right\} \) - rozšířená číselná osa

\bigskip

\noindent \textbf{Princip matematické indukce:}

Buď \( M \subset \mathbb{N} \) taková množina, že platí:
\begin{itemize}
    \item \( 1 \in M \)
    \item \( \forall n \in M : n + 1 \in M \)
\end{itemize}

Pak \( M = \mathbb{N} \)

\bigskip

\noindent \textbf{Definované operace s nekonečnem:}
\begin{itemize}
    \item \( \forall x \in \mathbb{R} : -\infty < x \land x < +\infty \)
    \item \( -\infty < +\infty \)
    \item \( \forall x > -\infty : x + (+\infty) = x + \infty = +\infty + x = +\infty \)
    \item \( \forall x < +\infty : x + (-\infty) = x - \infty = -\infty + x = -\infty \)
    \item \( \forall x \in \mathbb{R}^+ \cup \left\{+\infty \right\} : x . (+\infty) = +\infty . x = +\infty \)
    \item \( \forall x \in \mathbb{R}^+ \cup \left\{+\infty \right\} : x . (-\infty) = -\infty . x = -\infty \)
    \item \( \forall x \in \mathbb{R}^- \cup \left\{-\infty \right\} : x . (+\infty) = +\infty . x = -\infty \)
    \item \( \forall x \in \mathbb{R}^- \cup \left\{-\infty \right\} : x . (-\infty) = -\infty . x = +\infty \)
    \item \( \forall x \in \mathbb{R} : \frac{x}{+\infty} = \frac{x}{-\infty} = 0 \)
    \item \( |-\infty | = |+\infty | = +\infty \)
\end{itemize}

\newpage

\subsection{Horní odhad a maximum množiny}
Buď \( M \subset \mathbb{R}^* \). Každé číslo \( k \in \mathbb{R}^* \) takové, že \( \forall x \in M : x \leq k \), nazýváme \textbf{horním odhadem} množiny \( M \).

\bigskip

\noindent Existuje-li horní odhad množiny \( M \), ktorý je prvkem množiny \( M \), nazýváme jej \textbf{maximem} množiny \( M \) a značíme \( max M \).

\subsection{Dolní odhad a minimum množiny}
Buď \( M \subset \mathbb{R}^* \). Každé číslo \( l \in \mathbb{R}^* \) takové, že \( \forall x \in M : x \geq l \), nazýváme \textbf{dolním odhadem} množiny \( M \).

\bigskip

\noindent Existuje-li dolní odhad množiny \( M \), ktorý je prvkem množiny \( M \), nazýváme jej \textbf{minimem} množiny \( M \) a značíme \( min M \).

\subsection{Supremum a infimum}
Buď \( M \subset \mathbb{R}^* \). Číslo \( s \in \mathbb{R}^* \), pro něž platí:
\begin{itemize}
    \item \( \forall x \in M : x \leq s\) (tzn. že \( s \) je horním odhadem \( M \))
    \item \( (\forall k \in \mathbb{R}^* , k < s)(\exists x \in M) : x > k \) (tzn. že žádné číslo menší než \( s \) není horním odhadem \( M \))
\end{itemize}
nazýváme \textbf{supremem} množiny \( M - s = sup M \). Jinak řečeno \( sup M \) je nejmenším horním odhadem množiny \( M \).

\bigskip

\noindent Buď \( M \subset \mathbb{R}^* \). Číslo \( i \in \mathbb{R}^* \), pro něž platí:
\begin{itemize}
    \item \( \forall x \in M : x \geq i\) (tzn. že \( i \) je dolním odhadem \( M \))
    \item \( (\forall l \in \mathbb{R}^* , l > i)(\exists x \in M) : x < l \) (tzn. že žádné číslo větší než \( i \) není dolním odhadem \( M \))
\end{itemize}
nazýváme \textbf{infimem} množiny \( M - i = inf M \). Jinak řečeno \( inf M \) je největším dolním odhadem množiny \( M \).

\bigskip

\noindent \textbf{Omezené množiny:} \\
Množinu \( M \subset \mathbb{R}^* \) nazveme:
\begin{itemize}
    \item \textbf{shora omezenou}, je-li \( sup M < +\infty \)
    \item \textbf{zdola omezenou}, je-li \( inf M > -\infty \)
    \item \textbf{omezenou}, je-li současně shora i zdola omezená
    \item \textbf{neomezenou}, není-li omezená 
\end{itemize}

\bigskip

\noindent \textbf{Věta o supremu:} \\
Každá podmnožina \( \mathbb{R}^* \) má právě jedno supremum. Důsledkem má každá podmnožina \( \mathbb{R}^* \) má právě jedno infimum.
